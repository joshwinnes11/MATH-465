% Options for packages loaded elsewhere
\PassOptionsToPackage{unicode}{hyperref}
\PassOptionsToPackage{hyphens}{url}
%
\documentclass[
]{article}
\usepackage{amsmath,amssymb}
\usepackage{iftex}
\ifPDFTeX
  \usepackage[T1]{fontenc}
  \usepackage[utf8]{inputenc}
  \usepackage{textcomp} % provide euro and other symbols
\else % if luatex or xetex
  \usepackage{unicode-math} % this also loads fontspec
  \defaultfontfeatures{Scale=MatchLowercase}
  \defaultfontfeatures[\rmfamily]{Ligatures=TeX,Scale=1}
\fi
\usepackage{lmodern}
\ifPDFTeX\else
  % xetex/luatex font selection
\fi
% Use upquote if available, for straight quotes in verbatim environments
\IfFileExists{upquote.sty}{\usepackage{upquote}}{}
\IfFileExists{microtype.sty}{% use microtype if available
  \usepackage[]{microtype}
  \UseMicrotypeSet[protrusion]{basicmath} % disable protrusion for tt fonts
}{}
\makeatletter
\@ifundefined{KOMAClassName}{% if non-KOMA class
  \IfFileExists{parskip.sty}{%
    \usepackage{parskip}
  }{% else
    \setlength{\parindent}{0pt}
    \setlength{\parskip}{6pt plus 2pt minus 1pt}}
}{% if KOMA class
  \KOMAoptions{parskip=half}}
\makeatother
\usepackage{xcolor}
\usepackage[margin=1in]{geometry}
\usepackage{color}
\usepackage{fancyvrb}
\newcommand{\VerbBar}{|}
\newcommand{\VERB}{\Verb[commandchars=\\\{\}]}
\DefineVerbatimEnvironment{Highlighting}{Verbatim}{commandchars=\\\{\}}
% Add ',fontsize=\small' for more characters per line
\usepackage{framed}
\definecolor{shadecolor}{RGB}{248,248,248}
\newenvironment{Shaded}{\begin{snugshade}}{\end{snugshade}}
\newcommand{\AlertTok}[1]{\textcolor[rgb]{0.94,0.16,0.16}{#1}}
\newcommand{\AnnotationTok}[1]{\textcolor[rgb]{0.56,0.35,0.01}{\textbf{\textit{#1}}}}
\newcommand{\AttributeTok}[1]{\textcolor[rgb]{0.13,0.29,0.53}{#1}}
\newcommand{\BaseNTok}[1]{\textcolor[rgb]{0.00,0.00,0.81}{#1}}
\newcommand{\BuiltInTok}[1]{#1}
\newcommand{\CharTok}[1]{\textcolor[rgb]{0.31,0.60,0.02}{#1}}
\newcommand{\CommentTok}[1]{\textcolor[rgb]{0.56,0.35,0.01}{\textit{#1}}}
\newcommand{\CommentVarTok}[1]{\textcolor[rgb]{0.56,0.35,0.01}{\textbf{\textit{#1}}}}
\newcommand{\ConstantTok}[1]{\textcolor[rgb]{0.56,0.35,0.01}{#1}}
\newcommand{\ControlFlowTok}[1]{\textcolor[rgb]{0.13,0.29,0.53}{\textbf{#1}}}
\newcommand{\DataTypeTok}[1]{\textcolor[rgb]{0.13,0.29,0.53}{#1}}
\newcommand{\DecValTok}[1]{\textcolor[rgb]{0.00,0.00,0.81}{#1}}
\newcommand{\DocumentationTok}[1]{\textcolor[rgb]{0.56,0.35,0.01}{\textbf{\textit{#1}}}}
\newcommand{\ErrorTok}[1]{\textcolor[rgb]{0.64,0.00,0.00}{\textbf{#1}}}
\newcommand{\ExtensionTok}[1]{#1}
\newcommand{\FloatTok}[1]{\textcolor[rgb]{0.00,0.00,0.81}{#1}}
\newcommand{\FunctionTok}[1]{\textcolor[rgb]{0.13,0.29,0.53}{\textbf{#1}}}
\newcommand{\ImportTok}[1]{#1}
\newcommand{\InformationTok}[1]{\textcolor[rgb]{0.56,0.35,0.01}{\textbf{\textit{#1}}}}
\newcommand{\KeywordTok}[1]{\textcolor[rgb]{0.13,0.29,0.53}{\textbf{#1}}}
\newcommand{\NormalTok}[1]{#1}
\newcommand{\OperatorTok}[1]{\textcolor[rgb]{0.81,0.36,0.00}{\textbf{#1}}}
\newcommand{\OtherTok}[1]{\textcolor[rgb]{0.56,0.35,0.01}{#1}}
\newcommand{\PreprocessorTok}[1]{\textcolor[rgb]{0.56,0.35,0.01}{\textit{#1}}}
\newcommand{\RegionMarkerTok}[1]{#1}
\newcommand{\SpecialCharTok}[1]{\textcolor[rgb]{0.81,0.36,0.00}{\textbf{#1}}}
\newcommand{\SpecialStringTok}[1]{\textcolor[rgb]{0.31,0.60,0.02}{#1}}
\newcommand{\StringTok}[1]{\textcolor[rgb]{0.31,0.60,0.02}{#1}}
\newcommand{\VariableTok}[1]{\textcolor[rgb]{0.00,0.00,0.00}{#1}}
\newcommand{\VerbatimStringTok}[1]{\textcolor[rgb]{0.31,0.60,0.02}{#1}}
\newcommand{\WarningTok}[1]{\textcolor[rgb]{0.56,0.35,0.01}{\textbf{\textit{#1}}}}
\usepackage{graphicx}
\makeatletter
\def\maxwidth{\ifdim\Gin@nat@width>\linewidth\linewidth\else\Gin@nat@width\fi}
\def\maxheight{\ifdim\Gin@nat@height>\textheight\textheight\else\Gin@nat@height\fi}
\makeatother
% Scale images if necessary, so that they will not overflow the page
% margins by default, and it is still possible to overwrite the defaults
% using explicit options in \includegraphics[width, height, ...]{}
\setkeys{Gin}{width=\maxwidth,height=\maxheight,keepaspectratio}
% Set default figure placement to htbp
\makeatletter
\def\fps@figure{htbp}
\makeatother
\setlength{\emergencystretch}{3em} % prevent overfull lines
\providecommand{\tightlist}{%
  \setlength{\itemsep}{0pt}\setlength{\parskip}{0pt}}
\setcounter{secnumdepth}{-\maxdimen} % remove section numbering
\ifLuaTeX
  \usepackage{selnolig}  % disable illegal ligatures
\fi
\usepackage{bookmark}
\IfFileExists{xurl.sty}{\usepackage{xurl}}{} % add URL line breaks if available
\urlstyle{same}
\hypersetup{
  hidelinks,
  pdfcreator={LaTeX via pandoc}}

\author{}
\date{\vspace{-2.5em}}

\begin{document}

\section{Looking at the data}\label{looking-at-the-data}

Using a markdown like this is a great way to follow along as you read
from the \href{https://r4ds.hadley.nz}{R for Data Science} or Tidy
Modeling with R books. It allows you to write text (to keep notes) along
with your code. Here we are going to load the \texttt{palmerpenguins}
data into our environment:

\begin{Shaded}
\begin{Highlighting}[]
\FunctionTok{data}\NormalTok{(penguins)}
\end{Highlighting}
\end{Shaded}

Notice that the \texttt{penguins} data set now appears in the
Environment tab on the upper right of our screen. We can get a look at
the data using the \texttt{glimpse} function:

\begin{Shaded}
\begin{Highlighting}[]
\FunctionTok{glimpse}\NormalTok{(penguins)}
\end{Highlighting}
\end{Shaded}

\begin{verbatim}
## Rows: 344
## Columns: 8
## $ species           <fct> Adelie, Adelie, Adelie, Adelie, Adelie, Adelie, Adel~
## $ island            <fct> Torgersen, Torgersen, Torgersen, Torgersen, Torgerse~
## $ bill_length_mm    <dbl> 39.1, 39.5, 40.3, NA, 36.7, 39.3, 38.9, 39.2, 34.1, ~
## $ bill_depth_mm     <dbl> 18.7, 17.4, 18.0, NA, 19.3, 20.6, 17.8, 19.6, 18.1, ~
## $ flipper_length_mm <int> 181, 186, 195, NA, 193, 190, 181, 195, 193, 190, 186~
## $ body_mass_g       <int> 3750, 3800, 3250, NA, 3450, 3650, 3625, 4675, 3475, ~
## $ sex               <fct> male, female, female, NA, female, male, female, male~
## $ year              <int> 2007, 2007, 2007, 2007, 2007, 2007, 2007, 2007, 2007~
\end{verbatim}

If you want to inspect a little more closely, click the
\texttt{penguins} data in your Environment pane on the upper right side
of the screen.

What do all these variables mean? This data is well-documented, so we
can find out more using \texttt{?penguins}. Try typing that yourself in
the console, and see what happens. We'll get more into importing and
wrangling data later on.

One more note: the beginning of each code chunk looks like
\texttt{\{r\ chunk-name\}}. The first \texttt{r} just indicates that we
are using \texttt{R} code, but other languages like python are supported
as well! The \texttt{chunk-name} is important, too, because it tells you
where to look if you make errors!

\section{First Plot}\label{first-plot}

Use this section to follow along with
\href{https://r4ds.hadley.nz/data-visualize}{Chapter 1 of R4DS}. Make a
new chunk to run code from the chapter, and eventually you should make
this image:

\includegraphics{https://r4ds.hadley.nz/data-visualize_files/figure-html/unnamed-chunk-7-1.png}
\# Second Plot

\begin{Shaded}
\begin{Highlighting}[]
\FunctionTok{ggplot}\NormalTok{(}\AttributeTok{data=}\NormalTok{penguins,}
       \AttributeTok{mapping =} \FunctionTok{aes}\NormalTok{(}\AttributeTok{x=}\NormalTok{bill\_length\_mm, }\AttributeTok{y=}\NormalTok{body\_mass\_g),}
\NormalTok{       ) }\SpecialCharTok{+}
  \FunctionTok{geom\_point}\NormalTok{(}\AttributeTok{mapping =} \FunctionTok{aes}\NormalTok{(}\AttributeTok{color=}\NormalTok{sex, }\AttributeTok{shape=}\NormalTok{sex)) }\SpecialCharTok{+}
  \FunctionTok{geom\_smooth}\NormalTok{(}\AttributeTok{method=}\StringTok{"lm"}\NormalTok{)}\SpecialCharTok{+}
  \FunctionTok{labs}\NormalTok{(}
    \AttributeTok{title=}\StringTok{"Bill Length (mm) vs. Body Mass (g)"}\NormalTok{,}
    \AttributeTok{subtitle =} \StringTok{"Colored by Sex"}\NormalTok{,}
    \AttributeTok{x =} \StringTok{"Bill Length (mm)"}\NormalTok{, }\AttributeTok{y =} \StringTok{"Body Mass (g)"}
\NormalTok{  )}
\end{Highlighting}
\end{Shaded}

\begin{verbatim}
## `geom_smooth()` using formula = 'y ~ x'
\end{verbatim}

\begin{verbatim}
## Warning: Removed 2 rows containing non-finite outside the scale range
## (`stat_smooth()`).
\end{verbatim}

\begin{verbatim}
## Warning: Removed 11 rows containing missing values or values outside the scale range
## (`geom_point()`).
\end{verbatim}

\includegraphics{Penguins_files/figure-latex/second plot-1.pdf}

\end{document}
